\subsection{Data Flow --- Import and Export of Files} \

The Student Solver Tool allows the course staff coordinator to import necessary input files in Excel format. 
Once uploaded, the application processes these files and generates structured JSON files that represent course staff assignments and preferences.
These JSON files can be edited directly within the application, containing the entire solving stage of the workflow within the website.
In the previous, manual process, the course staff coordinator started and ended with Excel files.
\\
The input files represented the preferences and availability data from course staff, and the enrollment data of the courses being offered.
We handle this input by taking the Excel files into the website through three separate upload fields: Course Allocations, PLA Preferences, and TA Preferences.
Once the files are uploaded, the application processes them and generates three JSON files that represent the same data in a structured format.
The course staff coordinator can then do their solving entirely within the application, using the JSON editor to assign course staff to courses based on their preferences and availability.
\\
Once this process is complete, it is time to export the results back into Excel format. This takes the form of one final sheet, the Course Staff Assignments file.
This file contains each course, with the lists of assigned PLAs and TAs in corresponding columns. This file is automatically generated when the export button is clicked on the website.
This Assignments file can be directly sent to each course's professor and staff.
\\
We designed this import and export functionality in order to allow us to streamline the workflow of the course staff coordinator, while preserving the format of the input and output
of the previous workflow. Future iterations of the project may enhance the input and output workflow, allowing course staff and professors to directly interact with the application.
This would eliminate the need for Excel files entirely, and allow for data to be directly sent from user to application. The current workflow, however, is still a significant improvement
over the previous manual process, and this import-export functionality allows us to achieve this improvement.