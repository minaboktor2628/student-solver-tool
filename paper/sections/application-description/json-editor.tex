\subsection{JSON Editor} \

To enable the course staff coordinator to easily work with structured data, we implemented a JSON editor within the application.
Upon upload of the necessary excel input files, three JSON files are created. These files represent the assignments of course staff
to specific courses, and the preferences of PLAs and TAs. We use JSON to store this data because it is a widely supported, human-readable 
format that allows for easy manipulation and validation.
\\
The JSON editor is in the form of a common code editor interface, with line numbering, syntax highlighting, and error checking. This interface provides a familiar environment
for technical users, such as the course staff coordinator. The editor also includes auto-completion, which helps to provide options for fields and values, reducing the chance of errors and increasing speed.
The auto-completion also is able to suggest best fits for course staff, based on their preferences and availability. When solving the assignments, the course staff coordinator can at any time locate
a specific course, enter the PLAs or TAs field, and press ctrl+space to see a list of course staff that are both qualified for the course and available for the term.
These features help to ensure data integrity and make the solving process more efficient. In a test run of the application, the course staff coordinator was able to complete the solving process in
approximately one hour, a significant improvement over the previous manual process.
\\