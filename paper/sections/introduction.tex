\section{Introduction} % Introduce the problem (ta/pla scheduling issue) and for now our current goal (just creating the validator) - around one page 
Assigning student staff to courses in a large computer science department is a complex and constraint-heavy task.
Each academic term, Prof.~Ahrens, the department’s current SATA Coordinator,
must assign Student Assistants (SAs/PLAs), Teaching Assistants (TAs), and Graduate Learning Assistants (GLAs)
to courses based on multiple factors, including student abilities, student preferences, instructor requirements,
and various ad-hoc constraints.
While some constraints remain consistent across terms, others shift,
producing a dynamic assignment problem that resists simple codification.
At present, these assignments are managed manually through spreadsheets.
This process requires the coordinator to check assignments by hand,
ensuring that no constraint has been violated---a task that quickly becomes analogous to solving a large,
irregular sudoku puzzle.
In addition, staffing needs must be estimated based on enrollment projections and the availability of student staff,
often with incomplete information.
As both enrollment numbers and the pool of student staff increase,
this manual approach grows increasingly inefficient, error-prone,
and difficult to scale.
The \tool was developed as a first step toward addressing this challenge.
Rather than immediately attempting to replace the manual process with a full constraint solver,
the validator serves as a tool to analyze existing spreadsheet assignments and detect inconsistencies
or violations of defined rules.
In this way, it acts as a lightweight satisfiability checker:
given an assignment and a set of constraints, does the current solution hold?
While the initial iteration focuses on validation,
the underlying approach anticipates integration with more advanced methods such
as satisfiability solving and probabilistic reasoning to automate assignments in future iterations.
The goal of this project was to design and implement a validator that can operate on current
spreadsheet-based artifacts, provide transparent error feedback,
and lay the groundwork for eventual extensions, including guided debugging and automated constraint solving.
The remainder of this paper will present the background on course assignment challenges,
the design and implementation of the Ahrens Validator, and an evaluation of its effectiveness in supporting the SATA coordination process.
